\documentclass[a4paper,11pt]{article}
\usepackage[T1]{fontenc}
\usepackage[utf8]{inputenc}
\usepackage{lmodern}

\title{Solid Geometry}
\author{Leander Maben}

\begin{document}

\maketitle
\tableofcontents

\begin{abstract}
This article discusses basic solid geometry.
The purpose is to try out mathematics mode in latex.
\end{abstract}

\section{Cuboid}
\subsection{General}

If \emph{l} is the length of a cuboid.
If \emph{b} is the breadth of a cuboid.
If \emph{h} is the height of a cuboid.

\begin{itemize}

\item Volume:
\begin{equation}
\label{cbd v}
A=lbh
\end{equation}
\item Lateral Surface Area:
\begin{equation}
\label{cbd lsa}
LSA=2h(l+b)
\end{equation}
\item Area:
\begin{equation}
\label{cbd tsa}
TSA=2(lb+bh+lh)
\end{equation}
\end{itemize}
\subsection{Cube: A special case}
A cube is a cuboid which has all three dimensions equal:
\begin{itemize}
\item Volume:
from eq\ref{cbd v}
\begin{equation}
\label{cb v}
Volume=a^3 \forall a>0
\end{equation}
\item Lateral Surface Area:
from eq\ref{cbd lsa}
\begin{equation}
\label{cb lsa}
LSA=4a^2
\end{equation}
\item Area:
from eq\ref{cbd tsa}
\begin{equation}
\label{cb tsa}
TSA=6a^2
\end{equation}
\end{itemize}
\section{Sphere}
If \emph{r} is the length of a cuboid.
\begin{itemize}

\item Volume:
\begin{equation}
Volume=\frac{4}{3}*\Pi*r^2
\end{equation}
\item Surface Area:
\begin{equation}
SA=4*\Pi*r^2
\end{equation}

\end{itemize}

\end{document}
