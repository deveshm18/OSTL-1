\documentclass[a4paper,11pt]{IEEEtran}
\usepackage[T1]{fontenc}
\usepackage[utf8]{inputenc}
\usepackage{lmodern}

\title{Computer}
\author{Leander}

\begin{document}

\maketitle
\tableofcontents

\begin{abstract}
This article is on computers.
The purpose is to demonstrate IEEE template usage.
\end{abstract}

\section{Etymology}
According to the Oxford English Dictionary, the first known use of the word "computer" was in 1613 in a book called The Yong Mans Gleanings by English writer Richard Braithwait: "I haue [sic] read the truest computer of Times, and the best Arithmetician that euer [sic] breathed, and he reduceth thy dayes into a short number." This usage of the term referred to a human computer, a person who carried out calculations or computations. The word continued with the same meaning until the middle of the 20th century. During the latter part of this period women were often hired as computers because they could be paid less than their male counterparts.By 1943, most human computers were women.
The Online Etymology Dictionary gives the first attested use of "computer" in the 1640s, meaning "one who calculates"; this is an "agent noun from compute (v.)". The Online Etymology Dictionary states that the use of the term to mean "'calculating machine' (of any type) is from 1897." The Online Etymology Dictionary indicates that the "modern use" of the term, to mean "programmable digital electronic computer" dates from "1945 under this name; [in a] theoretical [sense] from 1937, as Turing machine"

\section{History}
\subsection{Pre 20 th century}
The abacus was initially used for arithmetic tasks. The Roman abacus was developed from devices used in Babylonia as early as 2400 BC. Since then, many other forms of reckoning boards or tables have been invented. In a medieval European counting house, a checkered cloth would be placed on a table, and markers moved around on it according to certain rules, as an aid to calculating sums of money. 
The slide rule was invented around 1620–1630, shortly after the publication of the concept of the logarithm. It is a hand-operated analog computer for doing multiplication and division. As slide rule development progressed, added scales provided reciprocals, squares and square roots, cubes and cube roots, as well as transcendental functions such as logarithms and exponentials, circular and hyperbolic trigonometry and other functions. Slide rules with special scales are still used for quick performance of routine calculations, such as the E6B circular slide rule used for time and distance calculations on light aircraft. 
\subsection{First Computing Device}
Charles Babbage, an English mechanical engineer and polymath, originated the concept of a programmable computer. Considered the "father of the computer",[16] he conceptualized and invented the first mechanical computer in the early 19th century. After working on his revolutionary difference engine, designed to aid in navigational calculations, in 1833 he realized that a much more general design, an Analytical Engine, was possible. The input of programs and data was to be provided to the machine via punched cards, a method being used at the time to direct mechanical looms such as the Jacquard loom. For output, the machine would have a printer, a curve plotter and a bell. The machine would also be able to punch numbers onto cards to be read in later. The Engine incorporated an arithmetic logic unit, control flow in the form of conditional branching and loops, and integrated memory, making it the first design for a general-purpose computer that could be described in modern terms as Turing-complete.[17][18]

The machine was about a century ahead of its time. All the parts for his machine had to be made by hand – this was a major problem for a device with thousands of parts. Eventually, the project was dissolved with the decision of the British Government to cease funding. Babbage's failure to complete the analytical engine can be chiefly attributed to political and financial difficulties as well as his desire to develop an increasingly sophisticated computer and to move ahead faster than anyone else could follow. Nevertheless, his son, Henry Babbage, completed a simplified version of the analytical engine's computing unit (the mill) in 1888. He gave a successful demonstration of its use in computing tables in 1906. 
\section{Programs}
\subsection{Stored Program Architecture}
This section applies to most common RAM machine–based computers.

In most cases, computer instructions are simple: add one number to another, move some data from one location to another, send a message to some external device, etc. These instructions are read from the computer's memory and are generally carried out (executed) in the order they were given. However, there are usually specialized instructions to tell the computer to jump ahead or backwards to some other place in the program and to carry on executing from there. These are called "jump" instructions (or branches). Furthermore, jump instructions may be made to happen conditionally so that different sequences of instructions may be used depending on the result of some previous calculation or some external event. Many computers directly support subroutines by providing a type of jump that "remembers" the location it jumped from and another instruction to return to the instruction following that jump instruction.

Program execution might be likened to reading a book. While a person will normally read each word and line in sequence, they may at times jump back to an earlier place in the text or skip sections that are not of interest. Similarly, a computer may sometimes go back and repeat the instructions in some section of the program over and over again until some internal condition is met. This is called the flow of control within the program and it is what allows the computer to perform tasks repeatedly without human intervention.

Comparatively, a person using a pocket calculator can perform a basic arithmetic operation such as adding two numbers with just a few button presses. But to add together all of the numbers from 1 to 1,000 would take thousands of button presses and a lot of time, with a near certainty of making a mistake. On the other hand, a computer may be programmed to do this with just a few simple instructions. The following example is written in the MIPS assembly language:

\subsection{Machine Code}
In most computers, individual instructions are stored as machine code with each instruction being given a unique number (its operation code or opcode for short). The command to add two numbers together would have one opcode; the command to multiply them would have a different opcode, and so on. The simplest computers are able to perform any of a handful of different instructions; the more complex computers have several hundred to choose from, each with a unique numerical code. Since the computer's memory is able to store numbers, it can also store the instruction codes. This leads to the important fact that entire programs (which are just lists of these instructions) can be represented as lists of numbers and can themselves be manipulated inside the computer in the same way as numeric data. The fundamental concept of storing programs in the computer's memory alongside the data they operate on is the crux of the von Neumann, or stored program, architecture. In some cases, a computer might store some or all of its program in memory that is kept separate from the data it operates on. This is called the Harvard architecture after the Harvard Mark I computer. Modern von Neumann computers display some traits of the Harvard architecture in their designs, such as in CPU caches.

While it is possible to write computer programs as long lists of numbers (machine language) and while this technique was used with many early computers,[104] it is extremely tedious and potentially error-prone to do so in practice, especially for complicated programs. Instead, each basic instruction can be given a short name that is indicative of its function and easy to remember – a mnemonic such as ADD, SUB, MULT or JUMP. These mnemonics are collectively known as a computer's assembly language. Converting programs written in assembly language into something the computer can actually understand (machine language) is usually done by a computer program called an assembler. 
\end{document}
